This report describes the design and construction of the Tachometer for Sheaff's Buddy's Boat (TFSBB), a replacement tachometer driver for a 1988 Celebrity Champion power boat. A tachometer is a device that measures the speed an engine is turning, typically measured in revolutions per minute (RPM).  The power boat, belonging to Stewart Harvey, has an analog gauge and failed analog driver circuitry. The tachometer being described in this report is intended to replace the failed circuitry.

The TFSBB is designed to work with a four cylinder, four stroke, gasoline engine. In a gasoline engine, the ignition coil connects to a distributor that provides spark to the individual spark plugs. Because each cylinder fires every other rotation, the coil fires twice per revolution on a four cylinder engine. Thus, the RPM of the engine can be determined by measuring the frequency of the ignition signal from the engine. %Due to the four cylinders of the engine, the RPM of the motor is half of the number of pulses measured on the ignition signal. The same concept can be applied to other engines that employ similar ignition techniques.


When it comes to similar products, there is no equivalent on the market. Most tachometers come pre-installed by the vehicle manufacturer, and are integrated into the vehicle. One can buy a stand-alone tachometer unit, which contains both the gauge and driver circuitry, or can buy ICs that convert frequency to a voltage. The TFSBB however, fills a gap in the market, being a custom replacement for broken tachometer driver circuitry. The TFSBB also uses a microcontroller to accomplish the task of additional functionality such as measuring oil pressure and temperature.

An additional feature of the tachometer is a detachable liquid crystal display (LCD) showing analog engine information such as temperature, pressure, and battery voltage. Additional engine information alerts the operator to potential engine problems. The specifications for the tachometer are given below and are extracted from the project contract found in Appendix~\ref{app:contract}.

The project specifications include a DC-DC converter to step the nominal 12V battery voltage down to 3.3V. The DC-DC converter must output within 5\% of the specified voltage with less than 100mV of ripple. The tachometer must also be able to drive the gauge needle between 1000RPM and 4000RPM with an accuracy of 100RPM. The project must be built on a printed circuit board (PCB), be a deliverable product, and be capable of showing the RPM, coolant temperature, oil pressure, and battery voltage. The DC-DC converter specification was altered during the design process to supply 5V instead of 3.3V. The voltage change is discussed in Section~\ref{sec:ps}. The chosen method for displaying coolant temperature, oil pressure, and battery voltage is an LCD.

Section~\ref{sec:break} includes a functional block diagram and discusses the high level operation of the blocks. Section~\ref{sec:det} delves into the design process and decisions in more detail. Section~\ref{sec:res} reviews the results of the tachometer and presents evidence to prove functionality. Finally, Section~\ref{sec:con} concludes the report with a brief discussion of the results and a review of the report.